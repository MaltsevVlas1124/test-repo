\contest{ІІ (районний/міський) етап 2014/15 навч. року}{Черкаська обл.}{23.11.2014}
\renewenvironment{problemAllDefault}[1]{\vspace{10mm}\par\begin{problem}{#1}{\stdinOrInputTxt}{\stdoutOrOutputTxt}{1 сек}{64 мегабайти}}{\end{problem}}

\subsection{ІІ (районний/міський) етап 2014/15 н.~р.}

Задачі доступні для дорішування (\EjudgeCkipoName, \href{https://ejudge.ckipo.edu.ua/cgi-bin/new-register?contest_id=46}{змагання \textnumero$\,$46}).

Так склалося, що на \texttt{ejudge.ckipo.edu.ua} починаючи с\'{а}ме з цього змагання запроваджене \emph{комбіноване введення-виведення}: більшість задач налаштовані так, що програма учасника може читати вхідні дані хоч з клавіатури, хоч зі вхідного текстового файлу \verb"input.txt" (але\nolinebreak[2] лише з чогось одного, а\nolinebreak[3] не\nolinebreak[2] поперемінно). Аналогічно, програма учасника може виводити результати хоч на\nolinebreak[3] екран, хоч у вихідний текстовий файл\nolinebreak[2] \verb"output.txt" (теж лише на/у щось одне).

% В\nolinebreak[3] усіх задачах даного змагання, програма може читати вхідні дані хоч з клавіатури, хоч зі вхідного файлу \verb"input.txt" (але\nolinebreak[2] лише з чогось одного, а\nolinebreak[3] не\nolinebreak[2] поперемінно). Аналогічно, програма може виводити результати хоч на\nolinebreak[3] екран, хоч у вихідний текстовий файл\nolinebreak[2] \verb"output.txt" (теж лише на/у щось одне).


% % % \vspace{-\baselineskip plus 1cm minus 1mm}
\input 201415-2-A-statement	\input 201415-2-A-tutorial

% % % \vspace{-1ex plus 1cm minus 1mm}
\input 201415-2-B-statement	\input 201415-2-B-tutorial

\input 201415-2-C-statement	\input 201415-2-C-tutorial

\input 201415-2-D-statement	\input 201415-2-D-tutorial


