\contest{Дистанційний тур ІІІ (обласного) етапу 2013/14 навч. року}{Черкаська обл.}{18.01.2014}
\renewenvironment{problemAllDefault}[1]{\vspace{10mm}\par\begin{problem}{#1}{Клавіатура (stdin)}{Екран (stdout)}{1 сек}{64 мегабайти}}{\end{problem}}


\subsection{Дистанційний тур ІІІ (обласного) етапу 2013/14 н.~р.}

\hspace*{\parindent}У 2013/14 навч.\nolinebreak[3] році III (обласний) етап у Черкаській області складався з двох турів, де один був частково дистанційним (учасники приїздили не\nolinebreak[3] до\nolinebreak[3] м.~Черкаси, а\nolinebreak[3] до\nolinebreak[3] своїх райцентрів) і проводився на задачах черкаських авторів. С\'{а}ме він і наведений у цьому збірнику. Цей тур позиціонувався одночасно і\nolinebreak[2] як\nolinebreak[2] змагальний, і\nolinebreak[2] як\nolinebreak[2] кваліфікаційний; тому рівень складності комплекта задач дещо нижчий, ніж зазвичай на~III~етапі.

Задачі доступні для дорішування (\EjudgeCkipoName, \href{https://ejudge.ckipo.edu.ua/cgi-bin/new-register?contest_id=15}{змагання \textnumero$\,$15}). 

2-й тур відбувався у~Черкасах, але\nolinebreak[2] на\nolinebreak[2] задачах інших авторів, іншій системі (ejudge, але\nolinebreak[2] не\nolinebreak[2] \verb"ejudge.ckipo.edu.ua"), та\nolinebreak[3] й\nolinebreak[3] про його дорішування авторам цього збірника нічого не\nolinebreak[3] відомо. Тому, він до збірника не~включений.


\vspace{-0.5\baselineskip minus 1cm}
\input 201314-3a-A-statement	\input 201314-3a-A-tutorial

\input 201314-3a-B-statement	\input 201314-3a-B-tutorial

\input 201314-3a-C-statement	\input 201314-3a-C-tutorial

\input 201314-3a-D-statement	\input 201314-3a-D-tutorial


