\documentclass[10pt,a5paper]{extarticle}

\usepackage{cmap}
% \usepackage[T2A]{fontenc}
% \usepackage[utf8]{inputenc}

\usepackage{fontspec}
\setmainfont{Arial}[SlantedFont=Calibri-Light,BoldSlantedFont=Calibri-Bold]
\setsansfont{Verdana}
% \setmainfont{Verdana}
% \setsansfont{Arial}
\setmonofont{Courier New}

% \makeatletter
% \let\@alph\ori@alph
% \let\@Alph\ori@Alph
% \addto\extrasukrainian{\let\@alph\ori@alph \let\@Alph\ori@Alph}
% \makeatother

\usepackage[metapost,truebbox,mplabels]{mfpic}
\usepackage{listcorr}
\usepackage{epigraph}
\usepackage{floatflt}
\usepackage{multicol}
\usepackage{graphicx}
\usepackage{amstext}
\usepackage{amsmath}
\usepackage{amssymb}
\usepackage{amsfonts}
\usepackage{alltt}
\usepackage{cmap}
\usepackage{color}
\usepackage{refcount}

% % % \usepackage{wasysym}
% % % \usepackage{hyperref}
% % % \hypersetup{
    % % % colorlinks,
    % % % citecolor=black,
    % % % filecolor=black,
    % % % linkcolor=black,
    % % % urlcolor=black
% % % }

% \twocolumn

% \usepackage[landscape,russian]{olymppp}

\usepackage[ukrainian]{olymppp}

\usepackage{anysize}

\newif{\ifisART}
\isARTfalse

\def\ARTskip#1{\ifisART\else#1\fi}

\def\dib#1{\,#1\discretionary{}{\mbox{$#1$}}{}\,}
\def\dibbb#1{\)\nolinebreak\hspace{0.0625em plus 0.25em}\(\dib{#1}\)\nolinebreak\hspace{0.0625em plus 0.25em}\(}

\newenvironment{exampleSimple}[2]{
    \ttfamily\obeylines\obeyspaces\frenchspacing
    \newcommand{\exmp}[2]{
        \begin{minipage}[t]{#1}\rightskip=0pt plus 1fill\relax##1\medskip\end{minipage}&
        \begin{minipage}[t]{#2}\rightskip=0pt plus 1fill\relax##2\medskip\end{minipage}\\
        \hline
    }
	\begin{small}
    \begin{tabular}{|l|l|}
        \hline
        \multicolumn{1}{|c|}{\bf\texttt{Вхідні дані}}&
        \multicolumn{1}{|c|}{\bf\texttt{Результати}}\\
        \hline
}{
    \end{tabular}
	\end{small}
}


\marginsize{17mm}{17mm}{0mm}{10mm}


\opengraphsfile{pics}

\begin{document}

\def\<{\leqslant}
\def\>{\geqslant}
\def\dib#1{\,#1\discretionary{}{\mbox{$#1$}}{}\,}
\def\op{\mathop{\rm op}\nolimits}
\def\opt{\mathop{\rm opt}}
\def\isdiv{\mathbin{\hbox to 0.25em{\hfill\hbox to 0 pt{\raisebox{0pt}{\hss$\cdot$\hss}}\hbox to 0 pt{\raisebox{-.6ex}{\hss$\cdot$\hss}}\hbox to 0 pt{\raisebox{.6ex}{\hss$\cdot$\hss}}\hspace{0.15em}\hfill}\,}}

\newlength{\myparindent}


\newlength{\mytemplen}
\newlength{\mytemplensecond}
\newlength{\mytemplenthird}
\newsavebox{\mypictbox}
\def\myrightfigure#1#2{%
\savebox{\mypictbox}{\noindent{}#2}%
\settowidth{\mytemplen}{\usebox{\mypictbox}}%
\settoheight{\mytemplenthird}{\usebox{\mypictbox}}%
\ifdim\mytemplen<0.8\textwidth%
\noindent%
\setlength{\mytemplensecond}{\textwidth}%
\addtolength{\mytemplensecond}{-\mytemplen}%
\addtolength{\mytemplen}{3pt}% ??? better to find alike standard len
\hspace*{\mytemplensecond}\usebox{\mypictbox}%
\par\vspace*{-0.5\baselineskip}\par%
\vspace*{-\mytemplenthird}
\vspace{-\parskip}
\hangindent=-\mytemplen
\hangafter=-#1
\else
\begin{center}
\usebox{\mypictbox}%
\par
\end{center}
% \vspace{-\baselineskip}
\fi%
}

\def\mytextandpicture#1#2{%
\setlength{\myparindent}{\parindent}%
\savebox{\mypictbox}{\noindent{}#2}%
\settowidth{\mytemplensecond}{\usebox{\mypictbox}}%
\setlength{\mytemplen}{\textwidth}%
\addtolength{\mytemplen}{-\mytemplensecond}%
\addtolength{\mytemplen}{-3mm}%
\noindent\mbox{}\hfill\parbox{\mytemplen}{\hspace*{\myparindent}#1}\hfill\hspace{2.5mm}\hfill\parbox{\mytemplensecond}{\usebox{\mypictbox}}\hfill\mbox{}\\
}

\def\myhrulefill{\vspace{12mm}\par\vspace*{-12mm}\par\hrulefill}

\contest{Динамічне програмування (дні Іллі Порубльова)}{Школа <<Бобра>> з програмування, Львів}{03.10.2018}

\tableofcontents

\sloppy

% % % \section{Теоретичний матеріал}

% % % \input DP_Th

% % % \section{Література}

% % % \input lit

\section{Задачі першого дня (ДП)}

Цей комплект задач доступний для on-line перевірки 
на сайті \verb"https://ejudge.ckipo.edu.ua/", змагання №~64.

Значна частина задач цього комплекту\nolinebreak[3] --- класичні, 
справжнє авторство яких встановити вже важко.
Такими є, зокрема, задачі
A\nolinebreak[3] (платфоми-базова),
D\nolinebreak[3] (MaxSum-базова),
I\nolinebreak[3] (банкомат-базова).
Деякі з істотних модифікацій цих задач розроблені 
укладачем цього комплекту І.~Порубльовим ---
зокрема, задача~C (аналіз того, як зміна вартості стрибків 
може змінити задачу~A). Ідея розширень серії підзадач 
давно відома, але послідовність задач D, F,~G,
де\nolinebreak[3] така потреба виникає просто і природньо, розроблялася укладачем
(конкретно в~G використано також ідею Є.~Поліщука).


%%% Авторами задачі~H (розподіл станцій по зонам) є В.~Челноков та Д.~Поліщук.



\newenvironment{problemAllDefault}[1]{\vspace{10mm}\par\begin{problem}{#1}{Вхідні дані}{Результати}{1 сек}{256 мегабайтів}}{\end{problem}}

\input A   % \input A-tut

\input B   % \input B-tut

\input C   % \input C-tut

\input D   % \input D-tut
           
\input E   % \input E-tut
           
\input F   % \input F-tut

\input G   % \input G-tut
           
\input H   % \input H-tut
           
\input I   % \input I-tut

\input J   % \input J-tut

% \section{Задачі другого дня (коли ДП недоречне)}
% \setcounter{problem}{0}

% Цей комплект задач доступний для on-line перевірки 
% на сайті \verb"https://ejudge.ckipo.edu.ua/", змагання №~65.

% Частина задач комплекту (A,~B,~C) спеціально розроблені так, щоб на перший погляд сильно нагадувати деякі задачі комплекту попереднього дня і тим провокувати застосування~ДП.
% Решта задач теж мають \emph{частину} властивостей, потрібних динпрогу, але кінець кінцем ДП виявляється або взагалі незастосовним, або недоречним.
% Авторами задачі~E (розподіл станцій по зонам) є В.~Челноков та Д.~Поліщук.



% \input A2   \input A2-tut

% \input B2   \input B2-tut

% \input C2   \input C2-tut

% \input D2   \input D2-tut

% \input E2   \input E2-tut

% \input F2   \input F2-tut


% % % \clearpage
% % % \tableofcontents

\closegraphsfile
\end{document}